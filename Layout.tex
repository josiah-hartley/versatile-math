\usepackage[svgnames,table]{xcolor}
%\usepackage{minibox}
\usepackage{amsmath,pgfplots}
\usepackage[shortlabels]{enumitem}
%\usepackage{tkz-euclide}
%\usetkzobj{all}
\usepackage{tikz}
\usetikzlibrary{decorations.fractals}
\usepackage{amssymb}
\usepackage{etoolbox}
%\usepackage{cancel}
\usepackage{array}
\usepackage{graphicx}
\usepackage{framed}
\usepackage{tcolorbox}
\tcbuselibrary{breakable,skins}
\usepackage{marginnote}
%\usepackage{verbatim}
\usepackage[style=2]{mdframed}
\usepackage[top=0.75in, bottom=0.75in, inner=0.75in, outer=2in, marginparwidth=1.25in, marginparsep=0.25in]{geometry}
\usepackage{titlesec}
\usepackage[official]{eurosym}
%\usepackage[Bjornstrup]{fncychap}

\usepackage{changepage}
\usepackage[T1]{fontenc}
\usepackage{lmodern}
%\usepackage{url}
\usepackage{comment}
\ifpdf
  \usepackage{pdfcolmk}
\fi
\usepackage{graphicx}
\usepackage[hidelinks]{hyperref}
%\usepackage[outline]{contour}
\usepackage{pifont}
\ifxetex
  \usepackage{fontspec}
\fi

\strictpagecheck
\addtocontents{toc}{\protect\thispagestyle{empty}}
\usepgfplotslibrary{fillbetween}
\usepackage{longtable}

%%%%Author Name in Table of Contents---------
\makeatletter
\let\@nodottedtocline\@dottedtocline
\patchcmd{\@nodottedtocline}{\hbox{.}}{\hbox{}}{}{}
\patchcmd{\@nodottedtocline}{\normalcolor #5}{\normalcolor}{}{}
\newcommand*\l@sectionsubtitle{\@nodottedtocline{1}{0em}{1.5em}}
\makeatother

\def\sectionsubtitle#1{%
    \addcontentsline{toc}{sectionsubtitle}{\protect\numberline{}\textit{#1}}%
}
%%%%------------------------------------------

\newcommand{\sol}{\marginnote{\bfseries Solution}}

%%%%Define Gaussian Plot for Stats Chapter
\pgfmathdeclarefunction{gauss}{2}{%
  \pgfmathparse{1/(#2*sqrt(2*pi))*exp(-((x-#1)^2)/(2*#2^2))}%
}
%%%%End Define Gaussian Plot

%%%%Drawing Fractal
\newcommand{\bedknobsthree}[1]{
    %\begin{center}
        \tikz [
            decoration = #1
        ] \draw decorate{ decorate{ decorate{ (0, 0) -- (0.35\linewidth, 0) }}};
    %\end{center}
}
%%%%Drawing Fractal

%%%%Begin Header Format
\nouppercaseheads
\copypagestyle{doc}{companion}
\makeevenhead{doc}{\textbf{\Large \thepage} \hspace{0.1in} \textbf{CHAPTER \thechapter} \hspace{0.1in} \leftmark}{}{}
\makeoddhead{doc}{}{}{SECTION \rightmark \hspace{0.1in} \textbf{\Large \thepage}}
\makeevenfoot{doc}{}{}{}
\makeoddfoot{doc}{}{}{}
%\makeheadrule{doc}{\textwidth}{0pt} %Uncomment to remove headrule

\copypagestyle{chapter}{doc}
\makeevenhead{chapter}{}{}{}
\makeoddhead{chapter}{}{}{}
\makeevenfoot{chapter}{\textbf{\Large \thepage}}{}{}
\makeoddfoot{chapter}{\hfill}{\hfill}{\textbf{\Large \thepage}}
\makeheadrule{chapter}{\textwidth}{0pt} %Uncomment to remove headrule

\setsecheadstyle{\Large\scshape\raggedright\bfseries\color{blue!30!black}}
\setsubsecheadstyle{\large\scshape\raggedright}
\setsubsubsecheadstyle{\normalsize\scshape\raggedright}
\setbeforesecskip{-1.5ex plus -.5ex minus -.2ex}
\setaftersecskip{1.3ex plus .2ex}
\setbeforesubsecskip{-1.25ex plus -.5ex minus -.2ex}
\setaftersubsecskip{1ex plus .2ex}
%%%%End Header Format

%%%%Begin Chapter Header Format
\usepackage{color,calc}
\newsavebox{\ChpNumBox}
\definecolor{ChapBlue}{rgb}{0.2,0.6,1}
\makeatletter
%\newcommand*{\thickhrulefill}{%
%\leavevmode\leaders\hrule height 1\p@ \hfill \kern \z@}
\newcommand*{\thickhrulefill}{\rule{1.0\textwidth}{5pt}}
\newcommand*\BuildChpNum[2]{%
\begin{tabular}[t]{@{}c@{}}
\makebox[0pt][c]{#1\strut} \\[.5ex]
\colorbox{ChapBlue}{%
\rule[-10em]{0pt}{0pt}%
\rule{1ex}{0pt}\color{black}#2\strut
\rule{1ex}{0pt}}%
\end{tabular}}
\makechapterstyle{BlueBox}{%
\renewcommand{\chapnamefont}{\large\scshape}
\renewcommand{\chapnumfont}{\Huge\bfseries}
\renewcommand{\chaptitlefont}{\raggedright\Huge\bfseries}
\setlength{\beforechapskip}{20pt}
\setlength{\midchapskip}{26pt}
\setlength{\afterchapskip}{40pt}
\renewcommand{\printchaptername}{}
\renewcommand{\chapternamenum}{}
\renewcommand{\printchapternum}{%
\sbox{\ChpNumBox}{%
\BuildChpNum{\chapnamefont\@chapapp}%
{\chapnumfont\thechapter}}}
\renewcommand{\printchapternonum}{%
\sbox{\ChpNumBox}{%
\BuildChpNum{\chapnamefont\vphantom{\@chapapp}}%
{\chapnumfont\hphantom{\thechapter}}}}
\renewcommand{\afterchapternum}{}
\renewcommand{\printchaptertitle}[1]{%
\usebox{\ChpNumBox}\hfill
\parbox[t]{\hsize-\wd\ChpNumBox-1em}{%
\vspace{\midchapskip}%
\thickhrulefill\par
\chaptitlefont ##1\par}}%
}
\chapterstyle{BlueBox}
%%%%End Chapter Header Format

%%%%Begin Title Page Definition
\newlength{\drop}
\newcommand*{\FSfont}[1]{%
  \fontencoding{T1}\fontfamily{#1}\selectfont}
\renewcommand*{\FSfont}[1]{}%    kills special font selections

\newcommand*{\rotrt}[1]{\rotatebox{90}{#1}}
\newcommand*{\rotlft}[1]{\rotatebox{-90}{#1}}
\newcommand*{\topb}{%
  \resizebox*{\unitlength}{\baselineskip}{\rotrt{$\}$}}}
\newcommand*{\botb}{%
  \resizebox*{\unitlength}{\baselineskip}{\rotlft{$\}$}}}
\newcommand*{\titleBC}{\begingroup
\begin{center}
\settowidth{\unitlength}{\textit{\Large COMMON APPLICATIONS OF MATHEMATICS}} 
{\color{LightGoldenrod}\topb} \\[\baselineskip]
\textcolor{Sienna}{\textit{\HUGE Versatile Mathematics}} \\[\baselineskip]
{\color{RosyBrown}\Large COMMON MATHEMATICAL APPLICATIONS} \\
{\color{LightGoldenrod}\botb}
\end{center}
\vfill
\begin{center}
{\LARGE\textbf{Josiah Hartley}}\\
\text{}\\
{\LARGE\textbf{Erum Marfani}}\\
\text{}\\
{\LARGE\textbf{Val Lochman}}\\
\text{}\\
{\LARGE\textbf{Evan Evans}}\\
\text{}\\
{\LARGE\textbf{Dina Yagodich}}\\
\text{}\\
{\LARGE\textbf{Larry Huff}}\\
\vfill
\includegraphics[height=0.5in]{Banner.jpg}\\
2015
\end{center}
\endgroup}
%%%%End Title Page Definition

%Makes labels on axes smaller
\pgfplotsset{
    every axis/.append style={font=\tiny},  
}
%

%%%%Begin Procedure Definition
\newtcolorbox{proc}[1]{colback=brown!5!white,colframe=brown!80!white,toprule=5pt,fonttitle=\large,title=\textbf{#1}}
%%%%End Procedure Definition

\def\fra#1#2{\vskip #2\noindent\textcolor{black!60!red}{\rule{\textwidth}{2pt}}}

%%%%Begin Try It Definition
\renewcommand{\pfbreakdisplay}{%
\ding{167}\quad\ding{167}\quad\ding{167}}

\newenvironment{try}[1][http://www.frederick.edu/]
{\checkoddpage
\begin{tcolorbox}[beforeafter skip=10pt,breakable,colframe=black!20,colback=green!10,sharp corners=all]
\ifoddpage{\marginnote{\href{#1}{\bedknobsthree {Koch snowflake} \textbf{TRY IT}}}}
\else{\marginnote{\href{#1}{\textbf{TRY IT} \bedknobsthree {Koch snowflake}}}}
\fi
}
{\fancybreak*{\pfbreakdisplay} \end{tcolorbox}}
%%%%End Try It Definition

%%%%Begin Formula Definition
\newenvironment{formula}[1]
{\begin{tcolorbox}[beforeafter skip=10pt,colframe=black!10,colback=black!10,sharp corners=all]

{\large\bfseries #1}\\

}
{\end{tcolorbox}}
%%%%End Formula Definition

%%%%Counter for examples; reset for each chapter/section?
\newcounter{ExampleCounter}
\setcounter{ExampleCounter}{1}

%%%%Begin Example Definition
\newenvironment{example}[2][https://dl.dropboxusercontent.com/u/28928849/Webpages/MathInSocietyVideos2.htm]
{\checkoddpage 
\ifoddpage
  { \marginnote{\href{#1}{\begin{tcolorbox}[enhanced,
frame code={\path [tcb fill frame,fill=black!60!red] (frame.south east) -- ([yshift=-2mm]frame.north east) -- ([xshift=-2mm]frame.north east) -- (frame.north west) -- ([yshift=2mm]frame.south west) -- ([xshift=2mm]frame.south west) -- cycle; \path[draw=black!60!red,fill=black!60!red,line width=2pt]([xshift=-3.2mm,yshift=-0.35mm]frame.north east) -- ([xshift=-1.043\textwidth,yshift=-0.35mm]frame.north west);},nobeforeafter,
    colbacktitle=white,
    colback=black!60!red,
    coltitle=white,
    coltext=white,interior hidden,
    left=0pt,right=0pt,top=0.5pt, bottom=0.5pt]
 \ \textbf{\centering EXAMPLE \theExampleCounter} \end{tcolorbox}}}[6mm]
  }
\else
  {\marginnote{\href{#1}{\begin{tcolorbox}[enhanced,
frame code={\path [tcb fill frame,fill=black!60!red] (frame.south west) -- ([yshift=-2mm]frame.north west) -- ([xshift=2mm]frame.north west) -- (frame.north east) -- ([yshift=2mm]frame.south east) -- ([xshift=-2mm]frame.south east) -- cycle;\path[draw=black!60!red,fill=black!60!red,line width=2pt]([xshift=3.2mm,yshift=-0.35mm]frame.north west) -- ([xshift=1.26\textwidth,yshift=-0.35mm]frame.north west);},nobeforeafter,
    colbacktitle=white,
    colback=black!60!red,
    coltitle=white,
    coltext=white,interior hidden,
    left=0pt,right=0pt,top=0.5pt, bottom=0.5pt]
 \ \textbf{\centering EXAMPLE \theExampleCounter} \end{tcolorbox}}}[6mm]}
\fi 
\begin{tcolorbox}[beforeafter skip=10pt,breakable,colframe=black!20,colback=yellow!10,sharp corners=all] 
{\large\bfseries #2}

}
{ \end{tcolorbox} \stepcounter{ExampleCounter}}
%%%%End Example Definition

%%%%Begin Exercises Definition
\newcounter{ProblemNumber}

\newcommand\pone[1]{ %One exercise per line
  \noindent
  \begin{minipage}[t]{\textwidth}
  \vspace{0.1in}
    {\large\bfseries\theProblemNumber .} #1
  \vspace{0.1in}
  \end{minipage}
  \stepcounter{ProblemNumber}
}

\newcommand\ptwo[1]{ %Two exercises per line
  \noindent
  \begin{minipage}[t]{0.48\textwidth}
  \vspace{0.1in}
    {\large\bfseries\theProblemNumber .} #1
  \vspace{0.1in}
  \end{minipage}
  \stepcounter{ProblemNumber}
}

\newcommand\pthree[1]{ %Three exercises per line
  \noindent
  \begin{minipage}[t]{0.31\textwidth}
  \vspace{0.1in}
    {\large\bfseries\theProblemNumber .} #1
  \vspace{0.1in}
  \end{minipage}
  \stepcounter{ProblemNumber}
}

\newcommand\pfour[1]{ %Four exercises per line
  \noindent
  \begin{minipage}{0.23\textwidth}
  \vspace{0.15in}
    {\large\bfseries\theProblemNumber .} #1
  \vspace{0.15in}
  \end{minipage}
  \stepcounter{ProblemNumber}
}

\newenvironment{exercises}
{ 
	\newgeometry{margin=0.75in}	
	\setcounter{ProblemNumber}{1}
	\noindent{\href{https://www.myopenmath.com/index.php}{\Huge\bfseries\color{blue!60!black} Exercises \thesection}}\\
	{\color{blue!60!black} \rule{\textwidth}{3pt}}
	
}
{ \setcounter{ProblemNumber}{1}
  \restoregeometry }
%%%%End Exercises Definition